%%%%%%%%%%%%%%%%%
% This is an sample CV template created using altacv.cls
% (v1.7.4, 30 July 2025) written by LianTze Lim (liantze@gmail.com). Compiles with pdfLaTeX, XeLaTeX and LuaLaTeX.
%
%% It may be distributed and/or modified under the
%% conditions of the LaTeX Project Public License, either version 1.3
%% of this license or (at your option) any later version.
%% The latest version of this license is in
%%    http://www.latex-project.org/lppl.txt
%% and version 1.3 or later is part of all distributions of LaTeX
%% version 2003/12/01 or later.
%%%%%%%%%%%%%%%%

%% Use the "normalphoto" option if you want a normal photo instead of cropped to a circle
% \documentclass[10pt,a4paper,withhyper,normalphoto]{altacv}

\documentclass[10pt,a4paper,withhyper]{altacv}
%% AltaCV uses the fontawesome5 and simpleicons packages.
%% See http://texdoc.net/pkg/fontawesome5 and http://texdoc.net/pkg/simpleicons for full list of symbols.

% Change the page layout if you need to
\geometry{left=1.25cm,right=1.25cm,top=1.5cm,bottom=1.5cm,columnsep=1.2cm}

% The paracol package lets you typeset columns of text in parallel
\usepackage{paracol}

\usepackage{hyperref}

% Change the font if you want to, depending on whether
% you're using pdflatex or xelatex/lualatex
% WHEN COMPILING WITH XELATEX PLEASE USE
% xelatex -shell-escape -output-driver="xdvipdfmx -z 0" sample.tex
\iftutex
  % If using xelatex or lualatex:
  \setmainfont{Roboto Slab}
  \setsansfont{Lato}
  \renewcommand{\familydefault}{\sfdefault}
\else
  % If using pdflatex:
  \usepackage[rm]{roboto}
  \usepackage[defaultsans]{lato}
  % \usepackage{sourcesanspro}
  \renewcommand{\familydefault}{\sfdefault}
\fi

% Change the colours if you want to
\definecolor{SlateGrey}{HTML}{2E2E2E}
\definecolor{LightGrey}{HTML}{666666}
\definecolor{DarkPastelRed}{HTML}{0b0b64} %450808
\definecolor{PastelRed}{HTML}{11119c} %8F0D0D
\definecolor{GoldenEarth}{HTML}{000000} %E7D192
\colorlet{name}{black}
\colorlet{tagline}{PastelRed}
\colorlet{heading}{DarkPastelRed}
\colorlet{headingrule}{GoldenEarth}
\colorlet{subheading}{PastelRed}
\colorlet{accent}{PastelRed}
\colorlet{emphasis}{SlateGrey}
\colorlet{body}{LightGrey}

% Change some fonts, if necessary
\renewcommand{\namefont}{\Huge\rmfamily\bfseries}
\renewcommand{\personalinfofont}{\footnotesize}
\renewcommand{\cvsectionfont}{\LARGE\rmfamily\bfseries}
\renewcommand{\cvsubsectionfont}{\large\bfseries}


% Change the bullets for itemize and rating marker
% for \cvskill if you want to
\renewcommand{\cvItemMarker}{{\small\textbullet}}
\renewcommand{\cvRatingMarker}{\faCircle}
% ...and the markers for the date/location for \cvevent
% \renewcommand{\cvDateMarker}{\faCalendar*[regular]}
% \renewcommand{\cvLocationMarker}{\faMapMarker*}


% If your CV/résumé is in a language other than English,
% then you probably want to change these so that when you
% copy-paste from the PDF or run pdftotext, the location
% and date marker icons for \cvevent will paste as correct
% translations. For example Spanish:
% \renewcommand{\locationname}{Ubicación}
% \renewcommand{\datename}{Fecha}


%% Use (and optionally edit if necessary) this .tex if you
%% want to use an author-year reference style like APA(6)
%% for your publication list
% \input{pubs-authoryear}

%% Use (and optionally edit if necessary) this .tex if you
%% want an originally numerical reference style like IEEE
%% for your publication list
\input{pubs-num}

%% sample.bib contains your publications
\addbibresource{sample.bib}

\begin{document}
\name{Dominik G. Eichhorn}
\tagline{} %Statistiker
%% You can add multiple photos on the left or right
%\photoR{2.8cm}{Globe_High}
\photoL{2.6cm}{avatar}

\personalinfo{%
  % Not all of these are required!
  \email{dominik.g.eichhorn@gmail.com}
  \phone{+49 177 4845132}
  \mailaddress{Wiesenstraße 14 b, 84030 Ergolding}
  \location{München}
  \homepage{dgeichhorn.github.io}
  \linkedin{dgeichhorn}
  %\github{dgeichhorn}
  %\orcid{0009-0005-0710-750X}
  %% You can add your own arbitrary detail with
  %% \printinfo{symbol}{detail}[optional hyperlink prefix]
  % \printinfo{\faPaw}{Hey ho!}[https://example.com/]

  %% Or you can declare your own field with
  %% \NewInfoFiled{fieldname}{symbol}[optional hyperlink prefix] and use it:
  %\NewInfoField{gitlab}{\faGitlab}[https://gitlab.com/]
  %\gitlab{your_id}
  %%
  %% For services and platforms like Mastodon where there isn't a
  %% straightforward relation between the user ID/nickname and the hyperlink,
  %% you can use \printinfo directly e.g.
  % \printinfo{\faMastodon}{@username@instace}[https://instance.url/@username]
  %% But if you absolutely want to create new dedicated info fields for
  %% such platforms, then use \NewInfoField* with a star:
  %\NewInfoField*{mastodon}{\faMastodon}
  %% then you can use \mastodon, with TWO arguments where the 2nd argument is
  %% the full hyperlink.
  %\mastodon{@username@instance}{https://instance.url/@username}
}

\makecvheader
%% Depending on your tastes, you may want to make fonts of itemize environments slightly smaller
% \AtBeginEnvironment{itemize}{\small}

%% Set the left/right column width ratio to 6:4.
\columnratio{0.4}

% Start a 2-column paracol. Both the left and right columns will automatically
% break across pages if things get too long.
\begin{paracol}{2}

\cvsection{Über mich}

%\begin{quote}
%``For me, work is most fun when you develop together new ideas and put them into practice.''
%end{quote}


\begin{quote}
``Am meisten Spaß macht mir das Arbeiten, wenn man gemeinsam datengetriebene Lösungen entwickelt und diese in die Praxis umsetzt -- turning insights into impact.''
\end{quote}

%\begin{quote}
%``insights improve''
%\end{quote}

\medskip

\cvsection{Stärken}

% Don't overuse these \cvtag boxes — they're just eye-candies and not essential. If something doesn't fit on a single line, it probably works better as part of an itemized list (probably inlined itemized list), or just as a comma-separated list of strengths.

\cvtag{Motivation}
\cvtag{Lernbereitschaft}
\cvtag{Analytisches Denken}
\cvtag{Problem Lösen}
\cvtag{Kommunikationsfähigkeit}
\cvtag{Teamfähigkeit}

%\cvtag{Highly motivated}
%\cvtag{Analytical thinker}
%\cvtag{Problem solver}
%\cvtag{Quick learner}
%\cvtag{Team player}


%\cvtag{Motivator \& Leader}

%\cvtag{Hard-working}
%\cvtag{Eye for detail}
%\cvtag{Eager to learn}

\divider\smallskip

\cvtag{Statistik}
%\cvtag{Statistical Analysis}
\cvtag{Machine Learning}
\cvtag{Data Science}
\cvtag{Optimierung}
\cvtag{Modellierung}

\smallskip

\cvsection{Programmieren}

\cvtag{R}
\cvtag{Python}
\cvtag{SQL}

%\cvskill{R}{5}
%\divider
%
%\cvskill{Python}{3.5}
%\divider
%
%\cvskill{SQL}{2}

\bigskip

\cvsectionmix{ZERTIFIKATE (Ausw.)}
%\cvsection{Certificates (Sel.)}

\cvevent{Machine Learning Specialization}{Coursera: DeepLearning.ai}{}{}

%\divider

\cvevent{Neural Networks \& Deep Learning}{Coursera: DeepLearning.ai}{}{}

%\divider

\cvevent{Improving Deep Neural Networks: Hyperparameter Tuning, Regularization \& Optimization}{Coursera: DeepLearning.ai}{}{}

%\divider

\cvevent{Practical Time Series Analysis}{Coursera: State University of New York}{}{}

%\divider

\cvevent{Survival Analysis in R}{Coursera: Imperial College London}{}{}


\cvsection{Publikationen}

Hausenblas, P., \textbf{Eichhorn, D.}, Brieden, A., Soppert, M., Steinhardt, C. (2025). \textit{Improving network dynamic pricing policies through off- line reinforcement learning}. OR Spectrum.
\url{https://link.springer.com/article/10.1007/s00291-025-00821-2}.


\newpage


\cvsection{Besondere Erfolge}

\cvachievement{\faTrophy}{Top 10\% im Masterstudium}{Abschluss unter den besten 10\% der Studierenden in meinem Studiengang basierend auf der Abschlussnote}

\divider

\cvachievement{\faTrophy}{Abiturpreis der Deutschen Mathematiker-Vereinigung (DMV)}{Auszeichnung für herausragende Leistungen im Mathematik-Abitur}

\bigskip

\cvsection{Sprachen}

\cvskillmixed{Deutsch}{Muttersprache}
%\divider

\cvskillmixed{Englisch}{Verhandlungssicher}
%\divider

\cvskillmixed{Französisch}{Grundkenntnisse}
%\divider

\cvskillmixed{Latein}{Großes Latinum}

%\cvskill{English}{5} %% Supports X.5 values.
%\divider

%\cvskill{French}{4}
%\divider

\bigskip

\cvsection{Hobbys}

\cvtag{Mannschaftsrudern} 
\cvtag{Radfahren} 
\cvtag{Kochen}
\cvtag{Reisen} 
\cvtag{Lesen} 
%\cvtag{Besuchen von Museen}


%\cvsection{Referees}
%
%% \cvref{name}{email}{mailing address}
%\cvref{Prof.\ Alpha Beta}{Institute}{a.beta@university.edu}
%{Address Line 1\\Address Line 2}
%
%\divider
%
%\cvref{Prof.\ Gamma Delta}{Institute}{g.delta@university.edu}
%{Address Line 1\\Address Line 2}


%% Switch to the right column. This will now automatically move to the second
%% page if the content is too long.
\switchcolumn

\vspace{7pt}

\cvsection{Ausbildung}

\cvevent{Promotion in Statistik}{Universität der Bundeswehr München}{Seit Mai 2022}{}
Offline Reinforcement Learning, asymptotische statistische Theorie 

\divider

\cveventgrade{M.Sc.\ Management \& Technology}{Note: 1.2}{Technische Universität München}{Okt 2019 -- Apr 2022}{}
Spezialisierung: Finanzen, Nebenfach: Informatik

\divider

\cveventgrade{Auslandssemester}{}{Université Libre de Bruxelles}{Sep 2020 -- Jan 2021}{}

\divider

\cveventgrade{B.Sc.\ Mathematik}{Note: 2.0}{Technische Universität München}{Okt 2015 -- Sep 2019}{}
Spezialisierung: Statistik

\divider

\cveventgrade{B.Sc.\ Wirtschaftswissenschaften}{Note: 1.7}{Goethe Universität Frankfurt a.M.}{Okt 2012 -- Nov 2016}{}

\divider

\cveventgrade{Abitur}{Note: 1.3}{Siebold Gymnasium Würzburg}{Sep 2004 -- Jun 2012}{}

\medskip

\cvsection{Praxiserfahrung}

\cveventjob{PwC -- Projektarbeit in ``Predictive Maintenance''}{}{Apr 2021 -- Jun 2021}{München}
Eigenständige Entwicklung eines Machine Learning Modells in R für eine Predictive Maintenance Anwendung und Vergleich mit mehre-\\ren AutoML-Lösungen

\divider

\cveventjob{Allianz SE -- Praktikum in ``Business Intelligence''}{}{Jun 2019 -- Aug 2019}{München}
\begin{itemize}
\item Konzeption und Implementierung eines automatisierten Monito-\\rings für Kleinflotten-Tarife in R
\item Durchführung zielgerichteter Datenanalysen in R (inkl. entspre-\\chender Visualisierungen) für das Senior Management und Schnittstellenpartner
\end{itemize}

\divider

\cveventjob{BayernLB -- Praktikum in ``Credit Portfolio Risk''}{}{Aug 2017 -- Okt 2017}{München}
Unterstützung beim Testen des neuen bankinternen Kreditport-\\foliomodells

%\begin{itemize}
%\item Validated the treatment of simulation fluctuations and temporal dependencies in determining sector sensitivities
%\end{itemize}


\newpage

\cvsection{Akademische Erfahrung}

\cvevent{Lehrstuhl für Statistik}{Universität der Bundeswehr München}{Mai 2022 -- Sep 2025}{}
Wissenschaftlicher Mitarbeiter mit folgenden Tätigkeiten:
\smallskip
\begin{itemize}
\item Eigenständiges Bearbeiten eines wissenschaftlichen Forschungsthemas im Rahmen einer Promotion
\item Aktives Mitwirken an Forschungsprojekten des Lehrstuhls
\item Verantworten des Übungsbetriebs für Bachelor- und Master- vorlesungen aus dem Bereich Statistik und Data Analytics und Halten der entsprechenden Übungen
\item Führen eines Teams von Tutoren
\item Betreuen von Abschlussarbeiten, u.a. in Kooperation mit CapGemini und der Versicherungskammer Bayern
\item Betreuen von Seminaren, u.a. Seminare in Kooperation mit der Versicherungskammer Bayern auf Vorstandsebene
\end{itemize}

\divider

\cvevent{Institut für Informatik}{Technische Universität München}{Okt 2021 -- Apr 2022}{}
Tutor für ``Data Analysis \& Visualization in R''

\divider

\cvevent{Lehrstuhl für Mathematische Modellierung biologischer Systeme}{Technische Universität München}{Okt 2018 -- Feb 2019}{}
Tutor für ``Applied Regression''

\divider

\cvevent{Lehrstuhl für Strategie und Organisation}{Technische Universität München}{Mär 2018 -- Jun 2018}{}
Wissenschaftliche Hilfskraft

\divider

\cvevent{Lehrstuhl für Internationale Wirtschaftspolitik}{Goethe Universität Frankfurt a.M.}{Sep 2014 -- Nov 2014}{}
Wissenschaftliche Hilfskraft

\divider

\cvevent{Lehrstuhl für Volkswirtschaftslehre, insb. Wirtschaftliche Entwicklung und Integration}{Goethe Universität Frankfurt a.M.}{Mär 2014 -- Sep 2014}{}
Tutor für ``Einführung in die Volkswirtschaftslehre''



% Unterschrift
\vspace{30pt}

\includegraphics[scale=1.9]{Unterschrift_zugeschnitten.jpg}

Ergolding, 18.09.2025



%\cvsection{Projects}
%
%\cvevent{Project 1}{Funding agency/institution}{}{}
%\begin{itemize}
%\item Details
%\end{itemize}
%
%\divider
%
%\cvevent{Project 2}{Funding agency/institution}{Project duration}{}
%A short abstract would also work.
%
%\medskip

%\cvsection{A Day of My Life}
%
%% Adapted from @Jake's answer from http://tex.stackexchange.com/a/82729/226
%% \wheelchart{outer radius}{inner radius}{
%% comma-separated list of value/text width/color/detail}
%\wheelchart{1.5cm}{0.5cm}{%
%  6/8em/accent!30/{Sleep,\\beautiful sleep},
%  3/8em/accent!40/Hopeful novelist by night,
%  8/8em/accent!60/Daytime job,
%  2/10em/accent/Sports and relaxation,
%  5/6em/accent!20/Spending time with family
%}

% use ONLY \newpage if you want to force a page break for
% ONLY the current column
%\newpage




%\cvsection{Publications}

%% Specify your last name(s) and first name(s) as given in the .bib to automatically bold your own name in the publications list.
%% One caveat: You need to write \bibnamedelima where there's a space in your name for this to work properly; or write \bibnamedelimi if you use initials in the .bib
%% You can specify multiple names, especially if you have changed your name or if you need to highlight multiple authors.
\mynames{Eichhorn/Dominik,
  Eichhorn/D.\bibnamedelimi G.}
%% MAKE SURE THERE IS NO SPACE AFTER THE FINAL NAME IN YOUR \mynames LIST

%\nocite{*}

%\printbibliography[heading=pubtype,title={\printinfo{\faBook}{Books}},type=book]

%\divider

%\printbibliography[heading=pubtype,title={\printinfo{\faFile*[regular]}{Journal Articles}},type=article]

%\divider

%\printbibliography[heading=pubtype,title={\printinfo{\faUsers}{Conference Proceedings}},type=inproceedings]


\end{paracol}


\end{document}
